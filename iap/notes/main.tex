\documentclass{article}
\usepackage{listings}
\usepackage{xcolor}

\lstset{
  language=C,
  basicstyle=\ttfamily\small,
  keywordstyle=\color{blue}\bfseries,
  commentstyle=\color{gray},
  stringstyle=\color{red},
  numbers=left,
  numberstyle=\small\color{gray}, % Font leggibile per i numeri
  stepnumber=1,
  numbersep=0pt, % I numeri restano attaccati al riquadro
  tabsize=4,
  showspaces=false,
  showstringspaces=false,
  breaklines=true,
  frame=single,
  rulecolor=\color{gray},   % Colore del bordo
  framexleftmargin=10pt,    % Spazio sufficiente per i numeri all'interno
  xleftmargin=0pt,          % Margine esterno al riquadro
  xrightmargin=0pt,         % Margine esterno al riquadro
  framesep=5pt,             % Spazio interno al riquadro
  framewidth=0.4pt,         % Spessore del bordo
  captionpos=b,
}

\begin{document}

\begin{lstlisting}[caption=Calcolo di Fibonacci, label=fibonacci]
int fib(int n) {
    int fib,      // current
        fib_1,    // previous and
        fib_2;    // previous previous iterations

    if (n <= 2) return 1;  // caso base

    // prepara considerando n=2
    fib = fib_1 = 1;

    for (int i = 3; i <= n; i++) {
        // prepara iterazione corrente
        fib_2 = fib_1;
        fib_1 = fib;
        fib = fib_1 + fib_2;
    }

    return fib;
}
\end{lstlisting}

\end{document}
